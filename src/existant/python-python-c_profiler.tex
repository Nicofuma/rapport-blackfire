\Python fournit trois modules\footnote{\url{https://docs.python.org/2/library/profile.html}} permettant d'analyser les performances d'un programme et de générer des rapports à partir des données collectées. Ils utilisent tous les trois la même technique, à savoir l'utilisation de \verb|sys.setprofile()|\footnote{\verb?sys.setprofile()? ayant la particularité d'être intégré directement dans le cœur de \Python et d'accepter n'importe quelle fonction de rappel, qu'elle soit écrite en \C ou en \Python - Voir \url{https://docs.python.org/2/library/sys.html#sys.setprofile} pour plus de détails} pour définir une fonction de rappel qui est appelée à chaque fois que l'interpréteur entre ou sort d'une fonction. 
L'événement \verb|CALL|\footnote{Lancé quand l'interpréteur entre dans une fonction} permet de récupérer le nom de la fonction et de relever la date\footnote{La date est relevée de la manière la plus précise possible, la résolution restant néanmoins dépendante du système sur lequel l'application tourne} au début de la fonction. Quand à lui, l'événement \verb|RETURN|\footnote{Lancé quand l'interpréteur sort d'une fonction} permet de relever la date en sortie de la fonction, puis de calculer la différence entre le début et la fin de cette dernière et de stocker le résultat.

De plus ces trois modules collectent les mêmes données, à savoir les temps exclusif\footnote{En excluant le temps passé dans les autres fonctions} et inclusif\footnote{En incluant le temps passé dans les autres fonctions} passés dans chaque fonction et le nombre de fois qu'elle a été appelée. De plus ils génèrent leur rapport dans le même format, à savoir \verb|pstat|\footnote{\url{https://docs.python.org/2/library/profile.html#module-pstats}}, mais ils ont néanmoins de grandes différences d'implémentation et ne servent pas le même usage : 

\begin{itemize}
\item \textbf{cProfile} est le module recommandé dans la majorité des cas, il s'agit d'une extension \C ayant un impact sur les performances relativement limité.
\item \textbf{profile} est un module écrit en pur \Python et ayant exactement la même interface que le \emph{cProfile}. Par contre son impact sur les performances est très important, mais étant écrit en \Python il est plus facile de le modifier et de l'étendre.
\item \textbf{hotshot} est un autre module écrit en \emph{C}. A la différence du \emph{cProfile} son objectif est d'avoir un impact sur les performances le plus faible possible, quitte à devoir passer plus de temps dans une phase de post-traitement. En revanche il n'est plus maintenu aujourd'hui et a été supprimé dans \Python 3.
\end{itemize}