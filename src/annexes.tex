\chapter{New Relic - Décoration des méthodes}
\label{app:newrelicdecorationmethodes}

\pythonfilelong{codes/new-relic_decoration_modules.py}
\begin{listing}[H]
  \caption{New Relic - Décoration des méthodes}
\end{listing}

\chapter{Compatibilité Python 2 / Python 3}

\chapter{Décorateur WSGI}
  \label{app:decorateurwsgi}
  \pythonfilelong{codes/probe/blackfire/wsgi_wrapper.py}
  \begin{listing}[H]
    \caption{Décorateur wsgi - blackfire/wsgi\_wrapper.py}
  \end{listing}
 
\chapter{Profil complet}
  \label{app:fullProfile}
  \textfilelong{codes/profile.dat}
  \begin{listing}[H]
    \caption{Exemple de profile généré par \Blackfire}
  \end{listing}
 
\chapter{Structures de données}
  \section{La pile d'appels}
    \label{app:pile_struct}
    \begin{listing}[H]
      \caption{Structures de données permettant de gérer la pile d'appels}
      \cfile{codes/pile_struct.c}
    \end{listing}
  \section{Le graphe d'appels}
    \label{app:graph_struct}
    \begin{listing}[H]
      \caption{Structures de données permettant de gérer le graphe d'appels}
      \cfile{codes/graph_struct.c}
    \end{listing}
 
\chapter{Définition d'un PyCFunctionObject}
  \label{app:PyCFunctionObject}
  \begin{listing}[H]
    \caption{Structures de données représentant une fonction \C en \Python}
    \cfile{codes/PyCFunctionObject.c}
  \end{listing}
 
\chapter{Arguments de fonctions}
  \section[Instrumentation]{Instrumentation d'une liste de fonction}
  \label{app:do_instrument}
    \vspace{10px}
    \cfilelong{codes/do_instrument.c}
    \begin{listing}[H]
      \caption{Instrumentation d'une liste de fonction définie dynamiquement}
    \end{listing}
  \section[Fonction C]{Récupération des arguments d'une fonction C}
  \label{app:get_arg_from_cfunction}
    \begin{listing}[H]
      \caption{Récupération des arguments d'une fonction définie en \C}
      \cfile{codes/get_arg_from_cfunction.c}
    \end{listing}
  \clearpage
  \section[Fonction Python]{Récupération des arguments d'une fonction Python}
  \label{app:get_arg_from_python}
    \vspace{10px}
    \cfilelong{codes/get_arg_from_python.c}
    \begin{listing}[H]
      \caption{Récupération des arguments d'une fonction définie en \Python}
    \end{listing}
    
\chapter{Dimension opcodes}
  \label{app:count_bytecode}
  \cfilelong{codes/count_bytecode.c}
  \begin{listing}[H]
    \caption{Calcul du nombre d'opcodes dans la ligne courante}
  \end{listing}