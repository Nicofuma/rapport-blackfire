\chapter{Compatibilité Python 2 / Python 3}
  \label{app:compat-py3}
  \label{app:compat-23}
Actuellement deux versions majeures de \Python sont maintenues : \Python 2, en version 2.7, et \Python 3, en version 3.4. \Python 3 a apporté de grands changements au langage avec, par exemple, un support par défaut de l'UTF-8, mais a au passage cassé la rétro-compatibilité avec \Python 2 ; particulièrement au niveau de l'API C.

Parmi les changement importants qui ont impacté \Blackfire on peut noter :
\begin{itemize}
\item La gestion des chaînes de caractères qui a été complètement réécrite : celles-ci deviennent toutes UTF-8
\item La gestion des classes qui a été changée en profondeur afin de faire disparaître le type spécifique aux classes déclarées en \emph{Python}.
\item La suppression du type \verb|PyInt| au profit du type \verb|PyLong| déjà présent
\end{itemize}

Concrètement, au niveau de \emph{Blackfire}, un ensemble de macros a été défini afin de manipuler ces objets sans avoir besoin de se préoccuper de la version de \Python utilisée.

Pour les chaînes de caractères, le travail avait déjà été effectué par \emph{Yappi} et se présente sous la forme des six macros suivantes, permettant respectivement de :
\begin{itemize}
\item Convertir une chaîne \Python en chaîne \emph{C}.
\item Vérifier si un object \Python est une chaîne de caractères.
\item Créer une chaîne \Python à partir d'une chaîne \emph{C}.
\item Créer une chaîne \Python à partir d'un format (équivalent du \verb|sprintf| pour les chaînes \emph{C}).
\item Récupérer la taille d'une chaîne \Python (équivalent du \verb|strlen|).
\end{itemize}

  \begin{listing}[H]
    \caption{Abstraction des chaînes de caractères entre \Python 2 et \Python 3}
\begin{ccode}
#ifdef IS_PY3K
#define PyStr_AS_CSTRING(s) PyBytes_AS_STRING(PyUnicode_AsUTF8String(s))
#define PyStr_Check(s) PyUnicode_Check(s)
#define PyStr_FromString(s) PyUnicode_FromString(s)
#define PyStr_FromFormatV(fmt, vargs) PyUnicode_FromFormatV(fmt, vargs)
#define PyStr_FromFormat(fmt, ...) PyUnicode_FromFormat(fmt, __VA_ARGS__)
#define PyStr_GetSize(s) PyUnicode_GET_SIZE(s)
#else // < Py3x
#define PyStr_AS_CSTRING(s) PyString_AS_STRING(s)
#define PyStr_Check(s) PyString_Check(s)
#define PyStr_FromString(s) PyString_FromString(s)
#define PyStr_FromFormatV(fmt, vargs) PyString_FromFormatV(fmt, vargs)
#define PyStr_FromFormat(fmt, ...) PyString_FromFormat(fmt, __VA_ARGS__)
#define PyStr_GetSize(s) PyString_GET_SIZE(s)
#endif
\end{ccode}
  \end{listing}
  
Pour les entiers, ça se présente aussi sous la forme de macros permettant de vérifier si un type \Python représente un entier et de convertir un entier \Python en entier \emph{C}.

  \begin{listing}[H]
    \caption{Abstraction des entiers entre \Python 2 et \Python 3}
\begin{ccode}
#ifdef IS_PY3K
#define PyNumeric_Check(op) PyLong_Check(op)
#define PyNumeric_AsULong(op) (uint64_t)PyLong_AsUnsignedLong(op)
#else // < Py3x
#include <intobject.h>
#define PyNumeric_Check(op) (PyInt_Check(op) || PyLong_Check(op))
#define PyNumeric_AsULong(op) (PyInt_Check(op) ? (uint64_t)PyInt_AS_LONG(op) : (uint64_t)PyLong_AsUnsignedLong(op))
#endif
\end{ccode}
  \end{listing} 
  
Pour la gestion des classes, la principale différence concerne la disparition du type \verb|PyClass|\footnote{\verb?PyClass? étendant \verb?PyType?, il est important de vérifier si l'objet est un \verb?PyClass? avant de vérifier si c'est un \verb?PyType?.} au profit du type \verb|PyType| qui, dans \Python 2, représentait déjà les classes définies en \emph{C}.

  \begin{listing}[H]
    \caption{Détection d'une classe en \Python 2 et \Python 3}
\begin{ccode}
#ifndef IS_PY3K
if (PyClass_Check(class)) {
	// Python Class
	// <do something>
} else
#endif
if (PyType_Check(class)) {
	// "Class" defined in C or any class with Python3
	// <do something>
}
\end{ccode}
  \end{listing}
  
\chapter{Gestion de projet}
% TODO Un tableau avec 3 colonnes : début/fin/commentaire
% dire ce qui a été fait pas à pas
% + expliquer l'organisation de l'équipe

Durant mon stage j'étais un membre à part entière de l'équipe produit de \emph{SensioLabs} et à ce titre j'ai pris part au processus agile utilisé. L'équipe produit fonctionne avec un processus agile adapté de \emph{Scrum}, les sprints ont une durée de deux semaines, et, l'équipe ayant des membres travaillant à distance (il y a notamment eu plusieurs personnes travaillant depuis les États-Unis entre Avril et Juin), le \emph{stand-up meeting} a lieu tous les jours à 16h afin que tout le monde puisse participer. Enfin les membres de l'équipe prennent en charge le support client pendant une journée à tour de rôle.

\noindent\begin{tabular}{|C{0.25\textwidth}|L{0.68\textwidth}|}
   \hline
\textbf{\textsc{Période}} & \multicolumn{1}{c|}{\textbf{\textsc{Travail effectué}}} \\
   \hline
\multicolumn{2}{|c|}{\textbf{Prise en main de \Blackfire}} \\
23 Février - 6 Mars & Prise en main de \emph{Blackfire}, écriture de tests fonctionnels pour la CLI \\
9 Mars - 13 Mars & Intégration de \Blackfire à \emph{CloudControl}\footnote{ \url{https://www.cloudcontrol.com/} - \url{http://blog.blackfire.io/blackfire-on-cloudcontrol.html}} \\
16 Mars - 24 Mars & Écriture de tests fonctionnels pour le SDK \\
25 Mars - 1 Avril & Finalisation de l'intégration \emph{CloudControl} \\
2 Avril - 29 Avril & Analyse de \emph{Magento} et \emph{Wordpress} \\
   \hline
\multicolumn{2}{|c|}{\textbf{Sonde \Python}} \\
4 Mai - 5 Mai & Analyse du fonctionnement de \emph{New Relic}\footnote{Voir section \vref{sec:new-relic}} \\
6 Mai - 18 Mai & Preuve de faisabilité écrite en \Python : Analyse de performances \\
19 Mai - 21 Mai & Preuve de faisabilité écrite en \Python : Intégration avec \Blackfire \\
22 Mai - 4 Juin & Preuve de faisabilité écrite en \Python : Wrapper WSGI \\
8 Juin - 10 Juin & Preuve de faisabilité écrite en \Python : \verb?sitecustomize.py?, compatibilité \emph{MacOS} et \Python 3 \\
11 Juin - 12 Juin & Prise en main de la création de module \Python via l'api \C \\
16 Juin - 22 Juin & Implémentation de la collecte de données en \C - gestion de la pile d'appel, gestion du graphe d'appel, intégration avec la partie \Python\\
23 Juin - 24 Juin & Optimisation de performance de la partie \emph{C}, gestion de la mémoire \\
25 Juin - 3 Juillet & Études des possibilités pour la distribution de la sonde \Python et pour la compiler sur les différents environnements supportés \\
6 Juillet - 7 Juillet & Intégration d'une dimension opcodes \\
8 Juillet - 21 Juillet & Arguments de fonction en \Python 2 \\
22 Juillet - 23 Juillet & Arguments de fonction en \Python 3 \\
24 Juillet - 6 Août & Support des arguments de fonction sous \emph{MacOS} (recherche et correction d'un buffer overflow) \\
   \hline
\end{tabular}

\chapter{New Relic - Décoration des méthodes}
\label{app:newrelicdecorationmethodes}

\pythonfilelong{codes/new-relic_decoration_modules.py}
\begin{listing}[H]
  \caption{New Relic - Décoration des méthodes}
\end{listing}

\chapter{Décorateur WSGI}
  \label{app:decorateurwsgi}
  \pythonfilelong{codes/probe/blackfire/wsgi_wrapper.py}
  \begin{listing}[H]
    \caption{Décorateur wsgi - blackfire/wsgi\_wrapper.py}
  \end{listing}
 
\chapter{Profil complet}
  \label{app:fullProfile}
  \vspace{-20px}
  \textfilelong{codes/profile.dat}
  \vspace{-20px}
  \begin{listing}[H]
    \caption{Exemple de profil généré par \Blackfire}
  \end{listing}
 
\chapter{Structures de données}
  \section{La pile d'appels}
    \label{app:pile_struct}
    \begin{listing}[H]
      \caption{Structures de données permettant de gérer la pile d'appels}
      \cfile{codes/pile_struct.c}
    \end{listing}
  \section{Le graphe d'appels}
    \label{app:graph_struct}
    \begin{listing}[H]
      \caption{Structures de données permettant de gérer le graphe d'appels}
      \cfile{codes/graph_struct.c}
    \end{listing}
 
\chapter{Définition d'un PyCFunctionObject}
  \label{app:PyCFunctionObject}
  \begin{listing}[H]
    \caption{Structures de données représentant une fonction \C en \Python}
    \cfile{codes/PyCFunctionObject.c}
  \end{listing}
 
\chapter{Arguments de fonctions}
  \section[Instrumentation]{Instrumentation d'une liste de fonction}
  \label{app:do_instrument}
  \label{app:do_instrumentation}
    \vspace{10px}
    \cfilelong{codes/do_instrument.c}
    \vspace{-20px}
    \begin{listing}[H]
      \caption{Instrumentation d'une liste de fonction définie dynamiquement}
    \end{listing}
  \section[Fonction C]{Récupération des arguments d'une fonction C}
  \label{app:get_arg_from_cfunction}
    \begin{listing}[H]
      \caption{Récupération des arguments d'une fonction définie en \C}
      \cfile{codes/get_arg_from_cfunction.c}
    \end{listing}
  \clearpage
  \section[Fonction Python]{Récupération des arguments d'une fonction Python}
  \label{app:get_arg_from_python}
    \vspace{10px}
    \cfilelong{codes/get_arg_from_python_name.c}
    \vspace{-20px}
    \begin{listing}[H]
      \caption{Récupération, par nom, des arguments d'une fonction définie en \Python}
    \end{listing}
    \clearpage
    \cfilelong{codes/get_arg_from_python_num.c}
    \vspace{-20px}
    \begin{listing}[H]
      \caption{Récupération, par index, des arguments d'une fonction définie en \Python}
    \end{listing}
    
\chapter{Dimension opcodes}
  \label{app:count_bytecode}
  \cfilelong{codes/count_bytecode.c}
  \vspace{-20px}
  \begin{listing}[H]
    \caption{Calcul du nombre d'opcodes dans la ligne courante}
  \end{listing}
