% ---------- Various ---------- %
\newcommand{\Python}{\emph{Python} }
\newcommand{\PHP}{\emph{PHP} }
\newcommand{\C}{\emph{C} }
\newcommand{\Blackfire}{\emph{Blackfire} }
\newcommand{\SensioLabs}{\emph{SensioLabs} }

\setlength{\parskip}{1em}

% ---------- Logos ---------- %
\newcommand{\logo}[2][height=1cm]
{
	\includegraphics[#1]{images/logo/#2}
}

\newcommand{\logoBF}[1][height=1cm]{\logo[#1]{blackfire_secondary_square_transparent}}
\newcommand{\logoSL}[1][height=1cm]{\logo[#1]{sensiolabs}}
\newcommand{\logoIMAG}[1][height=1cm]{\logo[#1]{logo_ensimag}}

% ---------- Bloc 'Note' ---------- %
\colorlet{shadecolor}{gray!35}   % you may try 'blue' here
\renewenvironment{note}[1][Note]{%
  \def\FrameCommand{\textcolor{shadecolor}{\vrule width 5pt} \hspace{6pt}}%
  \MakeFramed {\advance\hsize-\width \FrameRestore}\noindent\textbf{#1}\;\;~}%
{\endMakeFramed}

% ---------- Header/Footer ---------- %
\renewcommand{\headrulewidth}{0.5pt}
\renewcommand{\footrulewidth}{0.5pt}
  
\fancypagestyle{plain}{%
  %\fancyhf{}
  \renewcommand{\headrulewidth}{0.5pt}
  \renewcommand{\footrulewidth}{0.5pt}
  \fancyfoot[L]{
    \logoBF[height=0.5cm]
    \logoSL[height=0.5cm]
    \logoIMAG[height=0.5cm]
  }
  \fancyfoot[R]{Page \thepage}
  \fancyfoot[C]{Rapport de projet de fin d'études}
}
  
\fancypagestyle{front}{%
  %\fancyhf{}
  \renewcommand{\headrulewidth}{0.5pt}
  \renewcommand{\footrulewidth}{0.5pt}
  \fancyfoot[L]{
    \logoBF[height=0.5cm]
    \logoSL[height=0.5cm]
    \logoIMAG[height=0.5cm]
  }
  \fancyfoot[R]{Page \thepage}
  \fancyfoot[C]{Rapport de projet de fin d'études}
}

\fancypagestyle{main}{%
  %\fancyhf{}
  \renewcommand{\headrulewidth}{0.5pt}
  \renewcommand{\footrulewidth}{0.5pt}
  \fancyfoot[L]{
    \logoBF[height=0.5cm]
    \logoSL[height=0.5cm]
    \logoIMAG[height=0.5cm]
  }
  \fancyfoot[R]{Page \thepage\ sur \pageref{LastPage}}
  \fancyfoot[C]{Rapport de projet de fin d'études}
}

\fancypagestyle{part}{%
  \fancyhf{}
  \renewcommand{\headrulewidth}{0.0pt}
  \renewcommand{\footrulewidth}{0.5pt}
  \fancyfoot[L]{
    \logoBF[height=0.5cm]
    \logoSL[height=0.5cm]
    \logoIMAG[height=0.5cm]
  }
  \fancyfoot[R]{Page \thepage\ sur \pageref{LastPage}}
  \fancyfoot[C]{Rapport de projet de fin d'études}
}

\pagestyle{fancy}

% ---------- Sommaire ---------- %
\newcommand{\sommaire}{
  { % Groupe pour isoler le sommaire
    \renewcommand*{\contentsname}{Sommaire}
    \makeatletter
      \renewcommand{\cftpartformatpnum}[1]{\hfil}
      % adapted from section 9.2.2 of memoir manual 
    \makeatother

    % On affiche les points mais sans le gras
    \renewcommand{\cftchapterleader}{\cftdotfill{\cftchapterdotsep}}
    
     % Espacement entre les . par défaut pour les sections
    \renewcommand{\cftchapterdotsep}{4.5}

    % On réduit l'espacement autour des chapitres et des parties
    \setlength{\cftbeforepartskip}{1.5em} 
    \setlength{\cftbeforechapterskip}{0.3em}
    
    % On indente les chapitres par rapport aux partie
    \setlength{\cftchapterindent}{15px}
    
    % On affiche la table des matières mais avec uniquement les chapitres et le sniveaux supérieurs
    \setcounter{tocdepth}{0}
    \tableofcontents*
    \setcounter{tocdepth}{4}
  } % end of local group
}

% ---------- Coloration Syntaxique ---------- %
\definecolor{codebg}{rgb}{0.96,0.96,0.96}
\usemintedstyle{manni}

% ---- PHP ----
\newcommand{\phpfile}[2][startinline]
{
  \inputminted[fontsize=\footnotesize, autogobble,bgcolor=codebg,frame=lines,linenos,funcnamehighlighting,startinline,breaklines,#1]{php}{#2}
}
\newcommand{\phpfilelong}[2][startinline]
{
  \begin{mdframed}[linecolor=black, topline=true, bottomline=true,
  leftline=false, rightline=false, backgroundcolor=codebg,userdefinedwidth=\textwidth]
  \inputminted[fontsize=\footnotesize,autogobble,linenos,funcnamehighlighting,startinline,breaklines,#1]{php}{#2}
  \end{mdframed}
}
\newminted{php}{linenos,frame=lines,autogobble,breaklines,funcnamehighlighting,bgcolor=codebg,startinline}

% ---- Python ----
\newcommand{\pythonfile}[2][startinline]
{
  \inputminted[fontsize=\footnotesize, autogobble,bgcolor=codebg,frame=lines,linenos,funcnamehighlighting,startinline,breaklines,#1]{python}{#2}
}
\newcommand{\pythonfilelong}[2][startinline]
{
  \begin{mdframed}[linecolor=black, topline=true, bottomline=true,
  leftline=false, rightline=false, backgroundcolor=codebg,userdefinedwidth=\textwidth]
  \inputminted[fontsize=\footnotesize,autogobble,linenos,funcnamehighlighting,startinline,breaklines,#1]{python}{#2}
  \end{mdframed}
}
\newminted{python}{linenos,frame=lines,autogobble,breaklines,funcnamehighlighting,bgcolor=codebg,startinline}


% ---- C ----
\newcommand{\cfile}[2][startinline]
{
  \inputminted[fontsize=\footnotesize, autogobble,bgcolor=codebg,frame=lines,linenos,funcnamehighlighting,startinline,breaklines,#1]{c}{#2}
}
\newcommand{\cfilelong}[2][startinline]
{
  \begin{mdframed}[linecolor=black, topline=true, bottomline=true,
  leftline=false, rightline=false, backgroundcolor=codebg,userdefinedwidth=\textwidth]
  \inputminted[fontsize=\footnotesize,autogobble,linenos,funcnamehighlighting,startinline,breaklines,#1]{c}{#2}
  \end{mdframed}
}
\newminted{c}{linenos,frame=lines,autogobble,breaklines,funcnamehighlighting,bgcolor=codebg,startinline}

\newminted{bash}{frame=lines,autogobble,breaklines,funcnamehighlighting,bgcolor=codebg,startinline}
\newminted{yaml}{frame=lines,autogobble,breaklines,funcnamehighlighting,bgcolor=codebg,startinline}

\newcommand{\textfile}[2][startinline]
{
  \inputminted[fontsize=\footnotesize,autogobble,bgcolor=codebg,frame=lines,linenos,funcnamehighlighting,startinline,breaklines,#1]{text}{#2}
}
\newcommand{\textfilelong}[2][startinline]
{
  \begin{mdframed}[linecolor=black, topline=true, bottomline=true,
  leftline=false, rightline=false, backgroundcolor=codebg,userdefinedwidth=\textwidth]
  \inputminted[fontsize=\footnotesize,autogobble,linenos,funcnamehighlighting,startinline,breaklines,#1]{text}{#2}
  \end{mdframed}
}
\newminted{text}{linenos,frame=lines,autogobble,breaklines,funcnamehighlighting,bgcolor=codebg,startinline}

% ------------ Do not wrap paragraph
\widowpenalties 1 10000
\raggedbottom

% ------------ Vertial align in list of listings === list of figures
\renewcommand{\memappchapinfo}[4]{%
  \addtocontents{lol}{\protect\addvspace{10pt}}}

\renewcommand{\memchapinfo}[4]{%
  \addtocontents{lol}{\protect\addvspace{10pt}}}

% ------------ Linstings wording and numbering
\renewcommand\listoflistingscaption{Liste des codes}
\renewcommand\listingscaption{\textsc{Code}}
\counterwithin{listing}{section}
\renewcommand*{\thelisting}{\thepart.\thesection-\arabic{listing}}

% -------- Dans la liste des code la taille de la première colonne d'adapte à la taille du contenu
\makeatletter
\renewcommand{\numberline}[1]{%
  \@cftbsnum #1\@cftasnum~\@cftasnumb%
}
\makeatother

\makeatletter
  \@addtoreset{chapter}{part}
\makeatother

% ------------ Part toc
% http://web.mit.edu/texsrc/source/latex/minitoc/minitoc.sum
\renewcommand{\ptctitle}{Plan} 
\renewcommand{\thispageparttocstyle}{\thispagestyle{part}} 
%\noptcrule